%
% File ranlp2015.tex

%Contact iva@lml.bas.bg

%Based on style for ACL 2010 by
% jshin@csie.ncnu.edu.tw or pkoehn@inf.ed.ac.uk
%%
%% Based on the style files for ACL-IJCNLP-2009, which were, in turn,
%% based on the style files for EACL-2009 and IJCNLP-2008...

%% Based on the style files for EACL 2006 by 
%%e.agirre@ehu.es or Sergi.Balari@uab.es
%% and that of ACL 08 by Joakim Nivre and Noah Smith

\documentclass[11pt]{article}
\usepackage{ranlp2015}
\usepackage{times}
\usepackage{url}
\usepackage{latexsym}

\usepackage{mdwlist}
%\setlength\titlebox{6.5cm}    % You can expand the title box if you
% really have to

\title{Determining Environmental Context from Fictional Narratives}

\iffalse
\author{Leon\\
  University of Sheffield\\
  {\tt email1@domain1.com}  \And
  Nanna\\
  Aarhus University\\
  {\tt  email2@domain2.com}}
\fi

\date{}

\begin{document}
\maketitle
\begin{abstract}
Experiencing a narrative can be made more enjoyable and powerful by increasing the level of the immersion.
This can be achieved through contextual cues, such as sounds and images.
In order to provide these cues automatically, we attempt to detect parts of the environment described in a fictional narrative.
We present three machine learning approaches to this problem, compared to a rule-engineered baseline, and evaluate them over a dataset of fictional texts.
In addition, we analyse the performance of this system over the output of a text-to-speech system taking the same narratives as input.
This audio dataset is made available.
The system is able to recognise environmental context and respond appropriately.
\end{abstract}


\section{Introduction}

adding music changes mood, useful for immersion and positive experiences.
we will take narratives as input.
then, we'll work out the mood and appropriate environment music from text.
there is a diverse range of environments.
also, sfx for surprise events, like an approach horse, horses, thunder.
harder mode: work from text-to-speech.


~\cite{madden2009collaborative}
~\cite{huiberts2010captivating}

topic extraction~\cite{allan2002introduction}

\section{Method}

\subsection{Dataset}
For this dataset, we used paragraphs from a set of role-playing books~\cite{fangs,crypt,poe,sswamp}.
Each paragraph is set in a distinct location, making manual environment annotation simple and distinct.
The set of potential environmental items in this genre is:

\begin{itemize*}
\item Mountain
\item Hill
\item Forest
\item Swamp
\item Windy
\item Blizzard
\item Rain
\item Lightning
\item Stream
\item River
\item Campfire
\item Night
\item Meadow
\item Road
\item Town
\item Crowd
\item Tavern
\item Underground
\item Trotting horse
\item Galloping horse
\end{itemize*}

Paragraphs were labelled with one or more of these labels by a human annotator.
In total, X paragraphs were labelled.

\subsection{Spoken dataset}
An eventual use of this system is to provide automatic sound effects for a read story.
As a result, it should operate well on the output of a speech recognition system.
Paragraphs were read by an English native speaker, and recorded.
A speech recognition system~\cite{lamere2003design} then interpreted these readings and generated a textual representation for each paragraph.
The labels used in the text input corpus were then associated with these outputs.
This constitutes the spoken dataset.

\subsection{Baseline}
Trigger words of the name of the environment (swamp, rain, etc)

\subsection{Features and Classifier}
method: bag of n-grams + multiple nbayes; LD feature extraction~\cite{lui2011cross} + nbayes; bag of word reprs (w2v)~\cite{mikolov2013efficient} + multi-svm

\section{Results}

\section{Related Work}
KidsRoom~\cite{bobick1999kidsroom}. 

ML doc classification~\cite{sebastiani2002machine}.

NN good at binary doc classification~\cite{derczynski2006machine}.

SVM doc classification~\cite{isa2008text}.

\section{Conclusion}

\iffalse
\section*{Acknowledgments}
This project has received funding from the European Union’s Seventh Framework Programme for research, technological development and demonstration under grant agreement No. 611233, \textsc{Pheme}.
ah, was anything funding this? could anything be associated with this?? ahhh, erm
\fi

\bibliographystyle{acl}
\bibliography{rpg-scene}

\end{document}
